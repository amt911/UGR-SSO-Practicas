%TODO: ARREGLAR EJERCICIO 1B
\documentclass{article}
\usepackage[utf8]{inputenc}
\usepackage[spanish]{babel}
\usepackage{graphicx, graphics, float, hyperref}
\usepackage{listings}
\usepackage[a4paper, total={6in, 10in}]{geometry}

\title{SSO Práctica 1 Sesión 2}
\author{Andrés Merlo Trujillo}
\date{}
\hypersetup{
    colorlinks=true,
    linkcolor=black,
}

\begin{document}

\maketitle

\tableofcontents

\newpage
%\addcontentsline{toc}{section}{Ejercicio 1}
%\section*{Ejercicio 1}
%\begin{figure}[H]
%    \includegraphics[width=\textwidth]{imagenes/lsofi.png}
%\end{figure}

\addcontentsline{toc}{section}{Ejercicio 1}
\section*{Ejercicio 1}
Con la orden \verb|aa-status| o la orden \verb|apparmor_status| se pueden ver los perfiles activos en Ubuntu:

%foto de los perfiles
%caption: como se puede ver hay 42 perfiles cargados.

Ahora voy a elegir el perfil \verb|/usr/bin/man|, para poder ver el archivo del perfil asociado basta con irse al directorio \verb|/etc/apparmor.d| y el archivo se denomina igual que la ruta absoluta del mismo, pero en vez de usar ``/'' se utilizan puntos. Por tanto, el archivo deseado es: \verb|/etc/apparmor.d/usr.bin.man|.

%foto del archivo

%otra foto


Las compoenntes principales son las siguientes:

\begin{itemize}
    \item \verb|#include <tunables/global>| carga un archivo que contiene las definiciones de las variables.
    \item \verb|/usr/bin/man| Ruta absoluta del binario.
    \item \verb|#include <abstractions/base>| obtiene los componentes de los perfiles de AppArmor para simplifcar el desarrollo de perfiles.
    \item \verb|{,usr/}| Permite eliminar lineas innecesarias, poniendo los directorios similares dentro de la lista entre lalves. En este caso las opciones son \verb|/bin/bzip2| y \verb|/usr/bin/bzip2|
    \item \verb|... -> &man_groff| Utiliza el perfil referenciado en la derecha cuando \verb|man| utiliza algun comando de la izquierda.
    \item \verb|profile ... {| Perfiles secundarios que se ejecutarán cuando estos sean llamados desde el principal. Por ejemplo, mediante el enlace de otro comando desde el perfil principal cuando man lo llame.
\end{itemize}

Además, despues de la ruta de los comadnos, aparecen letras similares a los permisos de sistemas Linux, estos representan:

\begin{itemize}
    \item \textbf{r: }Modo lectura
    \item \textbf{w: }Modo escritura
    \item \textbf{a: }Modo adjuntar (append)
    \item \textbf{k: }Modo de bloqueo de archivo
    \item \textbf{l: }Modo de enlace
    \item \textbf{ux: }Modo de ejeccion sin restricciones
    \item \textbf{Ux: }Modo de ejeccion sin restricciones. Ademas, limpia el entorno (scrub the environment)
    \item \textbf{px: }Ejecucion discreta del perfil
    \item \textbf{Px: }MOdo de ejecucion discreta del perfil. Ademas, limpia el entorno (scrub the environment)
    \item \textbf{ix: }MOdo de ejecucion heredada
    \item \textbf{m: }Permite \verb|PROT_EXEC| con llamadas a \verb|mmap|
    \item \textbf{Cx: }Permite transiciones a un perfil hijo. Con la C mayuscula se usa ``secure exec'' de glibc.
\end{itemize}


\addcontentsline{toc}{section}{Ejercicio 2}
\section*{Ejercicio 2}
Voy a generar un perfil para el programa \verb|nano|, para saber su ruta absoluta se puede usar la orden \verb|which nano|:

%foto de which nano

Ahora para generar el perfil se ejecuta el comando \verb|aa-genprof /usr/bin/nano|:

%fotos

Ahora pide que abramos el programa a perfilar y pulsemos en el boton de escanear.

Es recomendable abrir el manual de capabilites para ver que significa cada capability con \verb|man 7 capabilities|.

%foto de dar_read_search

Esto permite saltarse las comoprobaciones de permisos de lecutra sobre el archivo y las comprobaciones sobre el directorio de permiso de lectura y ejecucion. Es mejor denegarlo con la tecla ``D''.

%foto nanorc
nanorc es un archivo con las configuraciones personlaizadas para el editor, al ser un archivo inofensivo se puede permitir su uso.



\end{document}

%\begin{figure}[H]
%    \includegraphics[width=\textwidth]{imagenes/passwdfile.png}
%    \caption{Ejemplo de entradas en el archivo.}
%\end{figure}