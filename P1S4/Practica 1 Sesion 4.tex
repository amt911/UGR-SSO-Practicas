%TODO: ARREGLAR EJERCICIO 1B
\documentclass{article}
\usepackage[utf8]{inputenc}
\usepackage[spanish]{babel}
\usepackage{graphicx, graphics, float, hyperref}
\usepackage{listings}
\usepackage[a4paper, total={6in, 10in}]{geometry}

\title{SSO Práctica 1 Sesión 3}
\author{Andrés Merlo Trujillo}
\date{}
\hypersetup{
    colorlinks=true,
    linkcolor=black,
}

\begin{document}

\maketitle

\tableofcontents

\newpage
%\addcontentsline{toc}{section}{Ejercicio 1}
%\section*{Ejercicio 1}
%\begin{figure}[H]
%    \includegraphics[width=\textwidth]{imagenes/lsofi.png}
%\end{figure}

\addcontentsline{toc}{section}{Ejercicio 1}
\section*{Ejercicio 1}

\addcontentsline{toc}{subsection}{Apartado A}
\subsection*{Apartado A}

Para crear las claves personales es necesario ejecutar la orden \verb|gpg --gen-key|. Ahora pedirá una serie de información que es necesario rellenar como el nombre o el correo.

%foto de nombre y apellidos.
\begin{figure}[H]
    \includegraphics[width=\textwidth]{imagenes/Captura desde 2022-10-19 16-42-45.png}
\end{figure}

A continuación aparece un sumario de los datos rellenados y al aceptarlos aparece un cuador de texto para introducir la contraseña. A continuacion pide al usuario realizar diversas cosas para generar entropia y que la clave sea lo mas aleatoria posible como mover el raton, realizar actividad de red y de disco, etc.

\begin{figure}[H]
    \includegraphics[width=\textwidth]{imagenes/Captura desde 2022-10-19 16-48-19.png}
\end{figure}

Y ahora para mostrar las claves publicas se utiliza el comando \verb|gpg --list-keys|:

%Captura desde 2022-10-19 16-55-18
\begin{figure}[H]
    \includegraphics[width=\textwidth]{imagenes/Captura desde 2022-10-19 16-55-18.png}
\end{figure}

Y para mostrar las claves privadas se utiliza el comando \verb|gpg --list-secret-keys|:

%Captura desde 2022-10-19 16-55-26
\begin{figure}[H]
    \includegraphics[width=\textwidth]{imagenes/Captura desde 2022-10-19 16-55-26.png}
\end{figure}

\addcontentsline{toc}{subsection}{Apartado B}
\subsection*{Apartado B}

Primero voy a crear un archivo de texto cualquiera como el siguiente:

%Captura desde 2022-10-19 16-57-15
\begin{figure}[H]
    \includegraphics[width=\textwidth]{imagenes/Captura desde 2022-10-19 16-57-15.png}
\end{figure}

Y ahora para cifrar el archivo se utiliza el comando \verb|gpg --armor --recipient correo --encrypt file|. Para ello, en la parte de ``recipient'' es necesairo poner el correo que se haya usado para crear las claves en el ejercicio anterior.

Esto generara un fichero con la extension ``.asc'' el cual, si se abre, contiene lo siguiente:

%Captura desde 2022-10-19 17-06-08
\begin{figure}[H]
    \includegraphics[width=\textwidth]{imagenes/Captura desde 2022-10-19 17-06-08.png}
\end{figure}


Por ultimo, para descifrar el mensaje se utiliza la orden \verb|gpg --decrypt prueba.asc|, se pedira la contraseña que se inserto al crear las claves.

%Captura desde 2022-10-19 17-07-30
\begin{figure}[H]
    \centering
    \includegraphics[width=0.6\textwidth]{imagenes/Captura desde 2022-10-19 17-07-30.png}
\end{figure}

%Captura desde 2022-10-19 17-07-50
\begin{figure}[H]
    \includegraphics[width=\textwidth]{imagenes/Captura desde 2022-10-19 17-07-50.png}
\end{figure}


\addcontentsline{toc}{subsection}{Apartado C}
\subsection*{Apartado C}

Me he juntado con el compañero \dots \textbf{RELLENAR AQUI ESTO} para mandarnos archivos cifrados a cada uno. Para que él pueda obtener mi clave publica, la he subido a \url{https://keyserver.ubuntu.com/}.

He creado el siguiente archivo para mi compañero:

%foto del archivo

Y con la orden \verb|gpg --armor --recipient correo --encrypt file| se encripta y se puede mandar el archivo con extension ``.asc'' sin problema.



Ahora, con el archivo cifrado de mi compañero, usando la orden \verb|gpg --decrypt file| se puede abrir.

%foto de la salida.


\addcontentsline{toc}{section}{Ejercicio 2}
\section*{Ejercicio 2}

Voy a cifrar el siguiente archivo:

%Captura desde 2022-10-19 17-33-08
\begin{figure}[H]
    \includegraphics[width=\textwidth]{imagenes/Captura desde 2022-10-19 17-33-08.png}
\end{figure}

El algoritmo de cifrado que voy a usar es ``desx'' y voy a usar 1234 iteraciones y con la contraseña ``qwerty''. Por tanto, para realizar esto, es necesario usar la orden \verb|openssl enc -des3 -iter 1234 -in ejercicio2 -out cifrado|

%Captura desde 2022-10-19 17-44-41
\begin{figure}[H]
    \includegraphics[width=\textwidth]{imagenes/Captura desde 2022-10-19 17-44-41.png}
\end{figure}

COmo se puede observar, pide la contraseña para encriptar, que es ``qwerty'' como he dicho antes. 

Ahora al mostrar el contenido del archivo cifrado, se ve que no es entendible:

%Captura desde 2022-10-19 17-44-49
\begin{figure}[H]
    \includegraphics[width=\textwidth]{imagenes/Captura desde 2022-10-19 17-44-49.png}
\end{figure}


Y ahora para descifrarlo, simplemente hace falta saber la contraseña, el algoritmo de cifrado y el numero de iteraciones. Todo esto se debe poner en el siguiente comando: \verb|openssl enc -des3 -iter 1234 -d -in cifrado|

%Captura desde 2022-10-19 17-54-02
\begin{figure}[H]
    \includegraphics[width=\textwidth]{imagenes/Captura desde 2022-10-19 17-54-02.png}
\end{figure}

\end{document}

%\begin{figure}[H]
%    \includegraphics[width=\textwidth]{imagenes/passwdfile.png}
%    \caption{Ejemplo de entradas en el archivo.}
%\end{figure}