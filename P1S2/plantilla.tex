%TODO: ARREGLAR EJERCICIO 1B
\documentclass{article}
\usepackage[utf8]{inputenc}
\usepackage[spanish]{babel}
\usepackage{graphicx, graphics, float, hyperref}
\usepackage{listings}
\usepackage[a4paper, total={6in, 10in}]{geometry}

\title{SSO Práctica 1 Sesión 2}
\author{Andrés Merlo Trujillo}
\date{}
\hypersetup{
    colorlinks=true,
    linkcolor=black,
}

\begin{document}

\maketitle

\tableofcontents

\newpage
\addcontentsline{toc}{section}{Ejercicio 1}
\section*{Ejercicio 1}

\addcontentsline{toc}{subsection}{Apartado A}
\subsection*{Apartado A}
Mediante la orden \verb|lsof -i| ejecutada como root, podemos obtener la informacion de los servicios y procesos que tienen alguna conexion abierta o archivo abierto.

%foto de lsof -i

La orden ofrece 9 columnas con los siguientes signifiacdos:

\begin{itemize}
    \item \textbf{COMMAND: }Nombre del comando asociado al proceso/archivo.
    \item \textbf{PID: }Process IDentificator (identificador de proceso).
    \item \textbf{USER: }UID del usuario al que pertenece el proceso/archivo.
    \item \textbf{FD: }Descriptor de fichero.
    \item \textbf{TYPE: }Tipo de archivo asociado al mismo (GDIR, GREG, ...) o indica el tipo de conexion (en cpaa de red) (IPv4, IPv6, X.25, etc.).
    \item \textbf{DEVICE: }Numero de dispotivio.
    \item \textbf{SIZE/OFF: }Tamaño del archivo.
    \item \textbf{NODE: }Numero de nodo/inodo de un fichero o el procotolo en capa de transporte (TCP, UDP, ...).
    \item \textbf{NAME: }Punto de montaje y sistema de archivos que usa el archivo abierto. Tambien puede significar la direccion local o remota de internet o de un socket.
\end{itemize}

A continuación explicaré dos procesos de la salida del comando anterior:

\addcontentsline{toc}{subsubsection}{sshd}
\subsubsection*{sshd}
\begin{itemize}
    \item \textbf{COMMAND: }sshd
    \item \textbf{PID: }1319
    \item \textbf{USER: }root
    \item \textbf{FD: }3u/4u (FDs 3 y 4. La letra ``u'' indica acceso de lectura y escritua)
    \item \textbf{TYPE: }IPv4/IPv6 (está a la espera de recibir algo en las dos versiones del protocolo IP.)
    \item \textbf{DEVICE: }22997/23008
    \item \textbf{SIZE/OFF: }0t0 (Offset, el segundo ``0'' indica que no hay offset)
    \item \textbf{NODE: }TCP (usan este protocolo de transporte porque asegura que se reciben los paquetes mediante ACKs).
    \item \textbf{NAME: }*:ssh (LISTEN) (El asterisco indica que espera de cualquier IP, en el puerto ssh (configurable, por defecto el 22)).
\end{itemize}

\addcontentsline{toc}{subsubsection}{avahi-daemon}
\subsubsection*{avahi-daemon}
\begin{itemize}
    \item \textbf{COMMAND: }avahi-dae (avahi-daemon)
    \item \textbf{PID: }1144
    \item \textbf{USER: }avahi
    \item \textbf{FD: }14u (FD 14. La letra ``u'' indica acceso de lectura y escritua)
    \item \textbf{TYPE: }IPv6
    \item \textbf{DEVICE: }22668
    \item \textbf{SIZE/OFF: }0t0 (Offset, el segundo ``0'' indica que no hay offset)
    \item \textbf{NODE: }UDP
    \item \textbf{NAME: }*:53167 (Cualquier IP en el puerto 53167).
\end{itemize}


\addcontentsline{toc}{subsection}{Apartado B}
\subsection*{Apartado B}
Leyendo el manual, hace falta usar el switch ``-i'', como en el apartado anterior, y añadiendo que busque las conexiones con el servicio ``ssh''. Por tanto, el comando quedaria asi: \verb|lsof -i :ssh|.

%foto del comando sin ssh desde host

Ahora mismo no hay nadie conectado, solo estan los ``daemons'' a la escucha de peticiones de conexion. Si ahora me conecto desde el otra maquina virtual a la de Ubuntu, la salida es la siguiente:

%foto del comando con ssh desde host.

Aparecen dos lineas nuevas y en el apartado \verb|NAME| se ve que la conexion es entre el usuario ``andres-kvm'' (Ubuntu) usando el servicio ``ssh'' (en mi caso es el puerto 22)  y el usuario ``archlinux'' en el puerto 57686, que es un puerto que se asigna aleatoriamente para enviar informacion (escuchar) a ``archlinux''.

Con la orden \verb|lsof -c sshd| se puede ver los archivos que tiene abiertos SSH:

%foto de alguno importante

Como se puede ver, aparece el usuario conectado y con el mismo PID aparecen todos los archivos abiertos por \verb|sshd|


\addcontentsline{toc}{subsection}{Apartado C}
\subsection*{Apartado C}
Para mostrar los archivos que usa un proceso concreto, es necesario referenciarlo con su PID. Para ello es necesario usar el siguiente comando: \verb|lsof -p PID|.

%foto de lsof -p 

Y ahora para ver los archivos que esta usando un usuario concreto, se debe usar el switch ``-u'': \verb|lsof -u usuario|

%foto de lsof -u
%caption: salida de lsof -u andres, se puede ver que en la tercera columna solo aparece ese usuario.

Por ultimo, para obtener los archivos que tiene abiertos un proceso \textbf{Y} un usuario, es necesario usar el switch adicional ``-a''. Esto es debido a que por defecto solo busca, en caso de haber varios switches, utilizando un criterio \textbf{OR}. Comando: \verb|lsof -u usuario -p PID -a|

%img de eso
%caption: archivos aboiertps por el PID PID y el usuario root

\addcontentsline{toc}{section}{Ejercicio 2}
\section*{Ejercicio 2}
\addcontentsline{toc}{subsection}{Apartado A}
\subsection*{Apartado A}
Para ver que vulnerabilidades hay en el sistema es neceasrio instalar el paquete \verb|lynis| junto al comando \verb|lynis audit system|.

%foto de los resultados

Y las posibles vulnerabilidades son las siguientes:

%foto de los warnings

Como se puede ver, solo hay dos avisos. Suponiendo que es una maquina para desarrollar aplicaciones, voy a listar los grados de severidad:

\begin{itemize}
    \item \textbf{Found one or more vulnerable packages. [PKGS-7392]} $\rightarrow$ Severidad: \textbf{Alta}. Puede llegar a ser muy peligroso, ya que pueden ser vulnerabilidades que potencialmente le otorguen acceso root al sistema. 
    
    \textbf{Solucion: }Para solucionarlo, es necesario actualizar todos los paquetes del sistema con la orden (en Ubuntu y en distros basadas en Debian) \verb|sudo apt upgrade|.


    \item \textbf{iptables module(s) loaded, but no rules active [FIRE-4512]} $\rightarrow$ Severidad: \textbf{Alta}. \verb|iptables| es un paquete que se utiliza principalmente junto a un firewall para permitir/bloquear cierto trafico. Si fuera una compañia importante sin firewall, podria darse el caso de que alguien entrase en el sistema y obtuviese datos sin permiso, produciendo asi un ``leak'' o incluso chantaje.
    
    \textbf{Solucion: }La solucion es habilitar el firewall y aplicarle las reglas que sean necesarias. En Ubuntu viene instalado por defecto \verb|ufw|, pero viene deshabilitado por defecto. Para habilitarlo hay que poner: \verb|sudo ufw enable| y con la orden \verb|sudo ufw status verbose| se pueden ver las reglas (por defecto prohibe trafico entrante y permite trafico saliente, prohibiendo asi conexiones del tipo SSH).

    %foto del status
\end{itemize}

Ahora, ejecutando de nuevo \verb|lynis audit system| aparece la siguiente puntuacion:

%puntuacion nueva (mas alta)

Y al ver los warnings se ve que no aparece ninguno:

%foto de great, no warnings

Por tanto, a nivel de advertencias el sistema ya está ``seguro'' (nunca se puede decir con seguridad). En cuanto a las sugerencias, las principales son para reforzar SSH y el uso de bloqueadores de IP como ``fail2ban''. No son fallos demasiado críticos.

\end{document}

%\begin{figure}[H]
%    \includegraphics[width=\textwidth]{imagenes/passwdfile.png}
%    \caption{Ejemplo de entradas en el archivo.}
%\end{figure}